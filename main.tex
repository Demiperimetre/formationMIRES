\documentclass[nopagenumber,9pt]{beamer}


\mode<presentation> {
  \usetheme[]{Berlin}
  \useoutertheme{shadow}
  \setbeamercovered{transparent}
  \usecolortheme{seahorse}
%\usecolortheme{sidebartab}
%  \usefonttheme{structurebold}
  \useinnertheme{default}
%\useinnertheme{rounded}
}
\usepackage{xcolor,pifont}
\usepackage{tikz}

\usetikzlibrary{calc,shapes,backgrounds,arrows,automata,shadows,positioning}

\usepackage{float}
\usepackage[french]{babel}
\usepackage{multicol}
\usepackage[utf8]{inputenc}
\usepackage{times}
\usepackage[T1]{fontenc}
%\usepackage{multirow}
\usepackage{color}
\usepackage{subfig} 
%variables vectorielles
\usepackage{amsmath, setspace, amsfonts, amssymb, graphics,multirow}
\newcommand{\I}{\mathbb{I}}
\newcommand{\E}{\mathbb{E}}
\renewcommand{\P}{\mathbb{P}}

\newcommand{\diag}{\mathop{\mathrm{diag}}}
\newcommand{\bbeta}{\boldsymbol{\beta}}
\newcommand{\balpha}{\boldsymbol{\alpha}}
\newcommand{\btheta}{\boldsymbol{\theta}}
\newcommand{\bY}{\mathbf{Y}}
\newcommand{\bX}{\mathbf{X}}
\newcommand{\bZ}{\mathbf{Z}}
\newcommand{\by}{\mathbf{y}}
\newcommand{\bz}{\mathbf{z}}
\newcommand{\ba}{\mathbf{a}}
\newcommand{\bt}{\mathbf{t}}
\newcommand{\bx}{\mathbf{x}}
\newcommand{\bh}{\mathbf{h}}
\newcommand{\bb}{\mathbf{b}}
\newcommand{\bB}{\mathbf{B}}
\newcommand{\bC}{\mathbf{C}}
\newcommand{\bM}{\mathbf{M}}
\newcommand{\bphi}{\boldsymbol{\phi}}
\newcommand{\blambda}{\boldsymbol{\lambda}}
\newcommand{\bepsilon}{\boldsymbol{\epsilon}}
\newcommand{\bpsi}{\boldsymbol{\psi}}
\newcommand{\bm}{\mathbf{m}}


\title
{I: Introduction aux réseaux et à leur analyse}%titre premiere page

%\subtitle{Presentation Soustitre}sp

\author{}



\date{
% Part of a joint work with 
% \begin{itemize}
%  \item Isabelle Goldringer (UMR de Génétique Végétale du Moulon),
% \item  Frédéric Hospital ,
% \item Stéphane Robin (UMR INRA 518).
%  \end{itemize}
% 
% \bigskip
Formation Analyse des réseaux\\
MIRES }

\subject{Séminaire}


\AtBeginSection[] {
 \begin{frame}<beamer>
   \frametitle{Outline}
   \tableofcontents[currentsection]
  \end{frame}
}



\AtBeginSubsection[] {
\begin{frame}<beamer>
   \frametitle{Plan}
   \tableofcontents[currentsection,currentsubsection]
 \end{frame}
}



\begin{document}

\begin{frame}
\titlepage

% \vspace{-.5cm}
% \includegraphics[scale=.1]{plots/AgroParisTech_-_logo.PNG}
% \vspace{-1.2cm}
% \begin{flushright}
%  \includegraphics[scale=.1]{plots/INRA_logo.jpg}
%  \end{flushright}

\end{frame}



\section{Premiers exemples et terminologie}

\begin{frame}
\frametitle{Données relationnelles}

\begin{center}
 \includegraphics[scale=.4]{plots/image_SBM.png}
\end{center}


\bigskip

Les réseaux peuvent représenter des
\begin{itemize}
\item relations sociales (amitié, connaissance, professionnelles),
\item échanges,
\item inventaires,
\item ...
\end{itemize}

\bigskip

Les réseaux peuvent \^etre ou non bipartites :  les interactions peuvent avoir lieu exclusivement entre des n\oe uds appartenant à deux groupes fonctionnels différents.


%illustration reseau

\end{frame}


\begin{frame}
 \frametitle{Terminologie}
 
 Un réseau est constitué de :
 \begin{itemize}
  \item n\oe uds / sommets qui représentent des individus / acteurs qui interagissent ou non,
  \item liens / ar\^etes / connexions qui représentent les interactions entre les paires de n\oe uds (dyad).
  
 \end{itemize}

\bigskip
 
Un réseau peut \^etre 
\begin{itemize}
  \item dirigé / orienté (e.g. échange),
  \item symétrique / non-dirigé (e.g. amitié),
  \item avec ou sans boucle.
 \end{itemize}

 Cette distinction a un sens uniquement pour les réseaux simples (pas bipartite).
 
 
\end{frame}



\begin{frame}
 \frametitle{Données disponibles et but}
 
 
\begin{center}
 \includegraphics[scale=.4]{plots/image_SBM.png}
\end{center}

\bigskip
\textbf{Données:}
\begin{itemize}
 \item  Le réseau est fourni par:
\begin{itemize}
 \item une matrice d'adjacence (réseau simple) ou une matrice d'incidence (réseau bipartite),
 \item une liste de dyades connectés (c'est-à-dire toutes les ar\^etes). Il est sous entendu que les dyades non mentionnés ne sont pas connectées...
\end{itemize}

\item des covariables additionnelles sur les n\oe uds ou sur les dyades.
 \end{itemize}



\bigskip


\textbf{Buts:}
\begin{itemize}
 \item Révéler / décrire / modéliser la topologie du réseau. 
 \item Découvrir des structures d'interactions particulières entre des sous-parties du réseau.
 \item Comprendre l'hétérogénéité du réseau.
 \item Pas d'inférer le réseau!
 \end{itemize}


\end{frame}




\begin{frame}
\frametitle{Représentation du réseau}

 \begin{columns}
 \begin{column}{.35\paperwidth}
Matrice d'adjacence : 
 $$X=\left(
\begin{array}{rrrrr}
0 & 1 & 0 & 0 \\ 
1 & 0 & 1 & 1 \\ 
0 & 0 & 0 & 0 \\ 
1 & 1 & 0 & 0 \\ 
\end{array}\right)
$$
\end{column}

\begin{column}{.2\paperwidth}
Liste d'ar\^etes (si dirigé attention à l'ordre...)
$$\left(
\begin{array}{rr}
1 & 2\\ 
1 & 4\\ 
2 & 3 \\ 
2 & 4\\ 
\end{array}\right)
$$ 
\end{column}


\begin{column}{.5\paperwidth}

\includegraphics[scale=.3]{plots/graphe_adj.pdf}

\end{column}

\end{columns}

\begin{itemize}
\item $n$ lignes et $n$ colonnes,
\item réseau non dirigé = matrice d'adjacence symétrique.
\end{itemize}

\end{frame}

\begin{frame}
\frametitle{Réseau bipartite}
 \begin{columns}
 \begin{column}{.35\paperwidth}
Matrice d'incidence
 $$X=\left(
\begin{array}{rrrrrrr}
  0 &   1 &   1 &   0 &   1 &   0 &   0 \\ 
  0 &   0 &   0 &   1 &   0 &   1 &   1 \\ 
  0 &   0 &   0 &   0 &   1 &   0 &   0 \\ 
  0 &   1 &   0 &   0 &   0 &   0 &   1 \\ 
\end{array}\right)
$$


\begin{itemize}
 \item n lignes et m colonnes, matrice rectangulaire.
 \item matrice d'adjacence correspondante $(n+m)\times(n+m)$:
 $$
 \left(
 \begin{array}{rr}
  0 & X\\
  X^T & 0
 \end{array}
 \right)
 $$
\end{itemize}

\end{column}

 \begin{column}{.2\paperwidth}
  Liste d'ar\^etes : 
 $$\left(
 \begin{array}{rr}
 A1 & B2 \\ 
 A1 & B3 \\ 
 A1 & B5 \\ 
 A2 & B4 \\ 
 A2 & B6 \\ 
 A2 & B7 \\ 
 A3 & B5 \\ 
 A4 & B2 \\ 
 A4 & B7 \\   
 \end{array}
 \right)$$ 
  
 \end{column}


\begin{column}{.3\paperwidth}

\includegraphics[scale=.25]{plots/graphe_bipartite.pdf}

\end{column}

\end{columns}



\end{frame}




\section{Représentation des réseaux}

\begin{frame}
 \frametitle{Différentes représentations possibles : au hasard}
 
 \begin{center}
 \includegraphics[scale=.3]{plots/karateRandom1.pdf}
  \includegraphics[scale=.3]{plots/karateRandom2.pdf}
  
 \end{center}
 
\end{frame}



\begin{frame}
 \frametitle{Fruchterman–Reingold}
 Ar\^etes de longueurs à peu près égales et peu de croisement. Mais représentations différentes à chaque appel.
 
 
  \begin{center}
 \includegraphics[scale=.3]{plots/karateFR1.pdf}
  \includegraphics[scale=.3]{plots/karateFR2.pdf}
  
 \end{center}
 
\end{frame}




\begin{frame}
 \frametitle{Autres représentations}
 
 \begin{columns}
  \begin{column}{.45\paperwidth}
   Graphe dirigé
   
   \centering
     \includegraphics[scale=.3]{plots/arbredirige.pdf}
  \end{column}
 \begin{column}{.45\paperwidth}
   Graphe bipartite
   
    \centering
     \includegraphics[scale=.3]{plots/graphe_bipartite.pdf}
  \end{column}
 \end{columns}

 
\end{frame}




\section{Statistiques résumées simples}

\begin{frame}
\frametitle{Quelques statistiques résumées simples}


% ref a mettre et sans doute à partager en plusieurs slides à la siute
\begin{itemize}

\item Degrés, sortant ou entrant si dirigés :
\begin{eqnarray*}
 \overrightarrow{D}_i&=&\sum_j X_{ij}\\
 \overleftarrow{D}_j&=&\sum_i X_{ij}
\end{eqnarray*}

\item Nestedness: si les n\oe uds de plus petit degré se connectent aux n\oe uds de plus haut degré,
\textcolor{blue}{Rodríguez-Gironés \& Santamaria (2006)}

\item Centralité (betweenness): pour un n\oe ud, nombres de chemins les plus courts entre n'importe quelle paire de n\oe uds passant par ce n\oe ud.
\textcolor{blue}{Freeman (1979)}

\item Modularité: mesure pour une partition de sa tendance à favoriser l'intra-connexion par rapport à l'inter-connexion.
$\Rightarrow$ Recherche de la meilleure partition. 
\textcolor{blue}{Clauset, Newman \& Moore (2004)}
\end{itemize}


\bigskip

Critères à adapter aux
\begin{itemize}
 \item réseaux dirigés,
 \item réseaux bipartites.
\end{itemize}

\bigskip

\textcolor{blue}{R packages: igraph, sna, vegan.} 


\end{frame}



\begin{frame}
 \frametitle{Example Chilean food web}
 
\begin{center}
\includegraphics[scale=.3]{plots/chilean_food_web.pdf} 
\end{center}

\begin{itemize}
 \item $n=106$ espèces / n\oe uds,
 \item densité: $12.1\%$.
\end{itemize}


\textcolor{blue}{Kéfi, Miele, Wieters, Navarrete \& Berlow (2016)}

\end{frame}


\begin{frame}
 \frametitle{Distribution des degrés}
 
 \begin{center}
\includegraphics[scale=.3]{plots/chilean_outdeg.pdf}  
\includegraphics[scale=.3]{plots/chilean_intdeg.pdf}
\end{center}

\end{frame}


\begin{frame}
 \frametitle{Nestedness}
 
 \begin{center}
  \includegraphics[scale=.3]{plots/chilean_nested.pdf}
 \end{center}


 \begin{itemize}
  \item plus généralement sur les graphes bipartites,
%  \item significance of the nestedness index computed by random permutations of the matrix,
  \item ce réseau est trouvé embo\^ité.
 
 \end{itemize}

 \end{frame}


\begin{frame}[fragile]
 \frametitle{Centralité (Betweenness)}
 
 \begin{center}
  \includegraphics[scale=.3]{plots/chilean_between.pdf}
 \end{center}

\begin{verbatim}
   Min. 1st Qu.  Median    Mean 3rd Qu.    Max. 
 0.000   0.000   0.000   6.604   6.929  59.570
\end{verbatim} 
 
\end{frame}


\begin{frame}
 \frametitle{Modularité}
\begin{center}
  \includegraphics[scale=.3]{plots/chilean_modularity.pdf}
 \end{center}
 
 
 
 \begin{itemize}
  \item
  \begin{tabular}{rrrrr}
  \hline
 1 & 2 & 3 & 4 \\ 
  \hline
  69 &  17 &   7 &  13 \\ 
   \hline
\end{tabular}
\item modularité très faible.
 \end{itemize}

 
\end{frame}



\section{Covariables}




\end{document}




\begin{frame}
\frametitle{A first random graph model for network: Null model}

\textcolor{blue}{Erd\H{o}s-Rényi (1959)} Model for $n$ nodes 

$$\forall 1\le i,j\le n,\quad X_{ij}\overset{i.i.d.}{\sim} b(p),$$
where $b$ is the Bernoulli distribution and $p\in[0,1]$ a probability for a link to exist. 


\begin{center}
 \includegraphics[scale=.3]{plots/ER.pdf} \includegraphics[scale=.3]{plots/degER.pdf}
\end{center}


\end{frame}


\begin{frame}
 \frametitle{Limitations of an ER graph to describe real networks}
 
 
 
\begin{itemize}
 \item Degree distribution too concentrated, no high degree nodes,
 \item all nodes are equivalent (no nestedness...),
 \item no modularity.

 \end{itemize}

\end{frame}






\section{Stochastic Block Model for classical networks}
%chilean web



\begin{frame}
  \frametitle{Stochastic Block Model}

  \begin{center}
    \begin{overlayarea}{\textwidth}{.5\textheight}
      \begin{columns}
        \begin{column}{.45\paperwidth}
        \begin{tikzpicture}
          %% UN GRAPH

          \tikzstyle{every edge}=[-,>=stealth',shorten >=1pt,auto,thin,draw]
          \tikzstyle{every state}=[draw=none,text=white,scale=0.65, font=\scriptsize, transform shape]
          \tikzstyle{every node}=[fill=yellow!40!orange]
          % premier cluster
          \node[state] (A1) at (0,0.5) {A1};
          \node[state] (A2) at (1,0.5) {A2};
          \node[state] (A3) at (.5,1.5) {A3};

          \path (A2) edge [bend left] node[fill=white,below=.1cm]
          {$\pi_{\textcolor{yellow!40!orange}{\bullet}\textcolor{yellow!40!orange}{\bullet}}$}
          (A1)
          (A1) edge [bend left] (A3)
          (A3) edge [bend left] (A2);

          \tikzstyle{every node}=[fill=blue!80!black]
          \foreach \angle/\text in {234/B1, 162/B2, 90/B3, 18/B4, -54/B5} {
            \node[fill=blue,state,xshift=5cm,yshift=3.5cm]     (\text)    at
            (\angle:1cm) {\text};
          }
          \path (B2) edge (B5)
          (B1) edge (B4);
          \foreach \from/\to in {1/2,2/3,4/5,5/1}{
            \path (B\from) edge [bend left] (B\to);
          }

          \path    (B3)    edge     [bend    left]    node[fill=white]
          {$\pi_{\textcolor{blue!80!black}{\bullet}\textcolor{blue!80!black}{\bullet}}$}  (B4) ;
          
          \tikzstyle{every node}=[fill=green!50!black]
          % troisieme cluster
          \node[state] (C1) at (3,-.5) {C1};
          \node[state] (C2) at (4,0) {C2};

          \path (C1) edge [bend right] node[fill=white,below=.25cm]
          {$\pi_{\textcolor{green!50!black}{\bullet}\textcolor{green!50!black}{\bullet}}$}
          (C2);

          % inter cluster
          \path (A3) edge [bend right]  (B2)
          (A3)    edge    [bend    left]    node[fill=white]
          {$\pi_{\textcolor{yellow!40!orange}{\bullet}\textcolor{blue!80!black}{\bullet}}$}
          (B3)
          (C2) edge [bend right] node[fill=white,right]
          {$\pi_{\textcolor{blue!80!black}{\bullet}\textcolor{green!50!black}{\bullet}}$}
          (B4)
          (A2) edge [bend right] node[fill=white]
          {$\pi_{\textcolor{yellow!40!orange}{\bullet}\textcolor{green!50!black}{\bullet}}$}
          (C1);
        \end{tikzpicture}
        \end{column}
        \begin{column}{.5\paperwidth}
          \begin{small}
            \begin{block}{Stochastic Block Model}
              Let $n$ nodes divided into
              \begin{itemize}
              \item
                $\mathcal{Q}=\{\textcolor{yellow!40!orange}{\bullet},\textcolor{blue!80!black}{\bullet},\textcolor{green!50!black}{\bullet}\}$
                classes
              \item  $\alpha_\bullet  =  \mathbb{P}(i  \in  \bullet)$,
                $\bullet\in\mathcal{Q},i=1,\dots,n$
              \item      $\pi_{\textcolor{yellow!40!orange}{\bullet}\textcolor{blue!80!black}{\bullet}}     =      \mathbb{P}(i
                \leftrightarrow j | i\in\textcolor{yellow!40!orange}{\bullet},j\in\textcolor{blue!80!black}{\bullet})$
              \end{itemize}
            \end{block}
          \end{small}
        \end{column}
      \end{columns}
    \end{overlayarea}
  \end{center}
  
%\begin{eqnarray*}
%&(Z_i) &  \ \sim^{\text{iid}} \mathcal{M}(1,\alpha) \ \text{et} \  Z_{i} \in \{1,...,Q\}, \\ 
% &(X_{ij})&| \ \{Z_{i},Z_{j}\} \sim^{\text{ind}} \mathcal{B}(\pi_{Z_{i}Z_{j}}).\\
%\end{eqnarray*}

% Proposition Julien
\begin{align*}
Z_i = \mathbf{1}_{\{i \in \bullet\}}  \ & \sim^{\text{iid}} \mathcal{M}(1,\alpha), \quad \forall\bullet \in \mathcal{Q}, \\ 
X_{ij} \ | \ \{i\in\textcolor{yellow!40!orange}{\bullet},j\in\textcolor{blue!80!black}{\bullet}\}
& \sim^{\text{ind}} \mathcal{B}(\pi_{\textcolor{yellow!40!orange}{\bullet}\textcolor{blue!80!black}{\bullet}})\\
\end{align*}

\end{frame}




\begin{frame}
\frametitle{Some remarkable structure generated with SBM : networks with hubs}

\centering
\begin{tabular}{ccc}
 \includegraphics[scale=.2]{plots/sbm/Etoile_reordered_adja_with_groups.png}&
\includegraphics[scale=.2]{plots/sbm/Etoile_graphe_with_colors.png}&
   \includegraphics[scale=.2]{plots/sbm/Etoile_graphe_resume.png}
 \end{tabular}

\begin{tabular}{cc}
    \includegraphics[scale=.2]{plots/sbm/Etoile_histogram_degree.png}&
   \includegraphics[scale=.2]{plots/sbm/Etoile_betweeness.png}
 \end{tabular}

\end{frame}



\begin{frame}
\frametitle{Some remarkable structure generated with SBM : community network}

\centering
\begin{tabular}{cc}
 \includegraphics[scale=.2]{plots/sbm/Affiliation_reordered_adja_with_groups.png}&
\includegraphics[scale=.2]{plots/sbm/Affiliation_graphe_with_colors.png} 
 \end{tabular}

\begin{tabular}{cc}
   \includegraphics[scale=.2]{plots/sbm/Affiliation_graphe_resume.png}
    \includegraphics[scale=.2]{plots/sbm/Affiliation_histogram_degree.png}&
 \end{tabular}

\end{frame}


\begin{frame}
\frametitle{Some remarkable structure generated with SBM : nestedness}

\centering
\begin{tabular}{cc}
 \includegraphics[scale=.2]{plots/sbm/Nested_reordered_adja_with_groups.png}&
\includegraphics[scale=.2]{plots/sbm/Nested_graphe_with_colors.png} 
 \end{tabular}

\begin{tabular}{cc}
   \includegraphics[scale=.2]{plots/sbm/Nested_graphe_resume.png}
    \includegraphics[scale=.2]{plots/sbm/Nested_histogram_degree.png}&
 \end{tabular}

\end{frame}




\begin{frame}
  \frametitle{Statistical inference}
 
    \begin{center}
  \begin{overlayarea}{\textwidth}{.5\textheight}
      \begin{columns}
        \begin{column}{.45\paperwidth}
        \begin{tikzpicture}
          %% UN GRAPH

          \tikzstyle{every edge}=[-,>=stealth',shorten >=1pt,auto,thin,draw]
          \tikzstyle{every state}=[draw=none,text=white,scale=0.65, font=\scriptsize, transform shape]
          \tikzstyle{every node}=[fill=lightgray]
          % premier cluster
          \node[state] (A1) at (0,0.5) {N1};
          \node[state] (A2) at (1,0.5) {N2};
          \node[state] (A3) at (.5,1.5) {N3};

          \path (A2) edge [bend left] node[fill=white,below=.1cm]
          {}
          (A1)
          (A1) edge [bend left] (A3)
          (A3) edge [bend left] (A2);

          \tikzstyle{every node}=[fill=blue!80!black]
          \foreach \angle/\text in {234/N1, 162/N2, 90/N3, 18/N4, -54/N5} {
            \node[fill=lightgray,state,xshift=5cm,yshift=3.5cm]     (\text)    at
            (\angle:1cm) {\text};
          }
          \path (B2) edge (B5)
          (B1) edge (B4);
          \foreach \from/\to in {1/2,2/3,4/5,5/1}{
            \path (B\from) edge [bend left] (B\to);
          }

          \path    (B3)    edge     [bend    left]    node[fill=white]
          {}  (B4) ;
          
          \tikzstyle{every node}=[fill=lightgray]
          % troisime cluster
          \node[state] (C1) at (3,-.5) {N1};
          \node[state] (C2) at (4,0) {N2};

          \path (C1) edge [bend right] (C2);

          % inter cluster
          \path (A3) edge [bend right]  (B2)
          (A3)    edge    [bend    left]    node[fill=white]
          {}
          (B3)
          (C2) edge [bend right] node[fill=white,right]
          {}
          (B4)
          (A2) edge [bend right] node[fill=white]
          {}
          (C1);
        \end{tikzpicture}
        \end{column}
        \begin{column}{.5\paperwidth}
          \begin{small}
            \begin{block}{Stochastic Block Model}
              Let $n$ nodes divided into
              \begin{itemize}
              \item
                $\mathcal{Q}=\{\textcolor{yellow!40!orange}{\bullet},\textcolor{blue!80!black}{\bullet},\textcolor{green!50!black}{\bullet}\}$,
                $\text{card}(\mathcal{Q})$ known
              \item  $\alpha_\bullet  =  ?$,
              \item      $\pi_{\textcolor{yellow!40!orange}{\bullet}\textcolor{blue!80!black}{\bullet}}     =      ?$
              \end{itemize}
            \end{block}
          \end{small}
        \end{column}
      \end{columns}
    \end{overlayarea}
    \end{center}
    \medskip

    
    \textcolor{blue}{Nowicki, \& Snijders (2001), Daudin et al. (2008)}
    
    \bigskip
    
\textcolor{blue}{R package: blockmodels.}
%     
%     \begin{thebibliography}{99}
%       \begin{scriptsize}
%       \bibitem[NS]{NS} Nowicki, Snijders, JASA, 2001 \newblock Estimation and prediction for
%         stochastic   blockstructures.
%         \textcolor{black}{} 
%       \bibitem[DRP]{DRP}   Daudin,  Picard,   Robin,  Statistics   and
%         Computing, 2008 \newblock A mixture model for random graphs. 
%       \end{scriptsize}
%   \end{thebibliography}

\end{frame}



\begin{frame}\frametitle{Statistical inference} 

From.... 

\centering
\begin{tabular}{cc}
 \includegraphics[scale=.2]{plots/sbm/Nested_adja.png}&
 \includegraphics[scale=.2]{plots/sbm/Nested_graphe_without_colors.png}
\end{tabular}
\end{frame}

\begin{frame}\frametitle{Statistical inference} 

... to 

\centering
\begin{tabular}{cc}
\includegraphics[scale=.2]{plots/sbm/Nested_reordered_adja_with_groups.png}
\includegraphics[scale=.2]{plots/sbm/Nested_graphe_with_colors.png}
\end{tabular}

\begin{block}{Statistician job}
\begin{itemize}
\item Find the clusters
\item Find the number of clusters
\item Practical implementation
\item Theoretical results
\end{itemize}
\end{block}

\end{frame}


\begin{frame}
 \frametitle{Application to the Chilean food web}
 
 \begin{center}
  \includegraphics[scale=.3]{plots/chilean_sbm.pdf}
 \end{center}

 \begin{itemize}
 \item $7$ groups/blocks/clusters found,
 \item \begin{tabular}{rrrrrrr}
  \hline
  1 & 2 & 3 & 4 & 5 & 6 & 7 \\ 
  \hline
  28 &  15 &  12 &  19 &  12 &  14 &   6 \\ 
   \hline
\end{tabular}
\end{itemize}


\end{frame}


\begin{frame}
 \frametitle{Application to the Chilean food web}
 
 \begin{center}
  \includegraphics[scale=.3]{plots/chilean_sbm_sum.pdf}
 \end{center}

 
 Another example for food web in \textcolor{blue}{Allesina \& Pascual (2009)}.
\end{frame}





\section{Latent Block Model for bipartite networks}
%reseau interaction qcq

\begin{frame}
\frametitle{Latent Block Model}

 \begin{center}
    \begin{overlayarea}{\textwidth}{.5\textheight}
      \begin{columns}
        \begin{column}{.45\paperwidth}
        \centering
        \includegraphics[scale=.3]{plots/LBM_exemple.pdf}
        \end{column}
        \begin{column}{.5\paperwidth}
          \begin{small}
            \begin{block}{Latent Block Model}
              \begin{itemize}
              \item
                $n$ row nodes $\mathcal{Q}_1=\{\textcolor{red}{\bullet},\textcolor{orange}{\bullet},\textcolor{green}{\bullet}\}$
                classes
              \item  $\alpha_\bullet  =  \mathbb{P}(i  \in  \bullet)$,
                $\bullet\in\mathcal{Q}_1,i=1,\dots,n$
              \item $m$ column nodes $\mathcal{Q}_2=\{\textcolor{yellow}{\bullet},\textcolor{black}{\bullet}\}$
                classes
               \item  $\beta_\bullet  =  \mathbb{P}(j  \in  \bullet)$,
                $\bullet\in\mathcal{Q}_2,j=1,\dots,m$
              \item      $\pi_{\textcolor{red}{\bullet}\textcolor{yellow}{\bullet}}     =      \mathbb{P}(i
                \leftrightarrow j | i\in\textcolor{red}{\bullet},j\in\textcolor{yellow}{\bullet})$
              \end{itemize}
            \end{block}
          \end{small}
        \end{column}
      \end{columns}
    \end{overlayarea}
  \end{center}
  
%\begin{eqnarray*}
%&(Z_i) &  \ \sim^{\text{iid}} \mathcal{M}(1,\alpha) \ \text{et} \  Z_{i} \in \{1,...,Q\}, \\ 
% &(X_{ij})&| \ \{Z_{i},Z_{j}\} \sim^{\text{ind}} \mathcal{B}(\pi_{Z_{i}Z_{j}}).\\
%\end{eqnarray*}

% Proposition Julien
\begin{align*}
Z_i = \mathbf{1}_{\{i \in \bullet\}}  \ & \sim^{\text{iid}} \mathcal{M}(1,\alpha), \quad \forall\bullet \in \mathcal{Q}_1, \\ 
W_j=\mathbf{1}_{\{j \in \bullet\}}  \ & \sim^{\text{iid}} \mathcal{M}(1,\beta), \quad \forall\bullet \in \mathcal{Q}_2, \\
X_{ij} \ | \ \{i\in\textcolor{red}{\bullet},j\in\textcolor{yellow}{\bullet}\}
& \sim^{\text{ind}} \mathcal{B}(\pi_{\textcolor{red}{\bullet}\textcolor{yellow}{\bullet}})\\
\end{align*}


\textcolor{blue}{Govaert \& Nadif (2008)} and 
\textcolor{blue}{R package: blockmodels} as well.

\end{frame}


\begin{frame}\frametitle{Incidence matrix point of view}

\centering
\begin{tabular}{cc}
 \includegraphics[scale=.2]{plots/lbm/Nested_adja.png}&
\includegraphics[scale=.2]{plots/lbm/Nested_reordered_adja_without_groups.png}
 \end{tabular}
 
 

\end{frame}



\begin{frame}
 \frametitle{LBM for ant-plant data}
 
 \begin{center}
  \includegraphics[scale=.3]{plots/bluthgen.pdf}
  \includegraphics[scale=.3]{plots/bluthgen_sum.pdf}
 \end{center}

 \begin{itemize}
  \item 2 blocks found over the $41$ ant species,
  \item 3 blocks found over the $51$ plant species.
 \end{itemize}

\textcolor{blue}{Blüthgen, Stork \& Fiedler (2004)}

\end{frame}




\section{Some possible extensions}


\begin{frame}
 \frametitle{Valued-edge networks or multiplex-edge networks}
 
 Information on edges can be something different from presence/absence.
 It can be:
 \begin{enumerate}
  \item a count of the number of observed interactions,
  \item a quantity interpreted as the interaction strength,
  \item several kind of interactions between nodes (Multiplex networks).
 \end{enumerate}

 \bigskip
 
 
Natural extensions of SBM and LBM for these three cases:
 \begin{enumerate}
  \item Poisson distribution: $X_{ij} \ | \ \{i\in\textcolor{yellow!40!orange}{\bullet},j\in\textcolor{blue!80!black}{\bullet}\}
\sim^{\text{ind}} \mathcal{P}(\lambda_{\textcolor{yellow!40!orange}{\bullet}\textcolor{blue!80!black}{\bullet}})$,
 \item Gaussian distribution: $X_{ij} \ | \ \{i\in\textcolor{yellow!40!orange}{\bullet},j\in\textcolor{blue!80!black}{\bullet}\}
\sim^{\text{ind}} \mathcal{N}(\mu_{\textcolor{yellow!40!orange}{\bullet}\textcolor{blue!80!black}{\bullet}},\sigma^2)$,
\textcolor{blue}{Mariadassou et al. (2010)}
\item Bivariate Bernoulli: $(X_{ij},X'_{ij} )\ | \ \{i\in\textcolor{yellow!40!orange}{\bullet},j\in\textcolor{blue!80!black}{\bullet}\}
\sim^{\text{ind}} \mathcal{B}^ 2(\pi_{\textcolor{yellow!40!orange}{\bullet}\textcolor{blue!80!black}{\bullet}})$. \textcolor{blue}{
Kefi et al. (2016), Barbillon et al. (2016)} 
 \end{enumerate}

 \bigskip
\textbf{Remark:} a particular case of multiplex network is dynamic network, \textcolor{blue}{Matias \& Miele (2015)}.
 
\end{frame}


\begin{frame}
 \frametitle{Multipartite networks}

 
  
 \begin{itemize}
  \item LBM is for bipartite networks,
  \item When there are more than two functional groups involved in interactions $\Rightarrow$ Multipartite networks.

  \item  Incidence matrix
 $$X=\left(
 \begin{array}{c|c|c}
  X_1&X_2&X_3
 \end{array}
 \right)\,,$$
 where
 \begin{itemize}
  \item $X_1, X_2, X_3$ correspond to the bipartite networks with the same functional groups in rows,
  \item for instance, $X_1$ is plant-pollinator network, $X_2$ is plant-ant network and $X_3$ is plant-seed dispersers network.
 \end{itemize}

  \item Extension of LBM quite natural but choice of the number of blocks is more challenging.
  
 \end{itemize}


 


\end{frame}


% 
% \begin{frame}
% \begin{itemize}
% \item Valued edges: abundance count, weighted interactions... % ecrire de sbouts de modele
% \item multiple interactions between nodes,
% \item multipartite networks: plants, pollinator, seed dispersers, ants...
% \item Taking into account sampling conditions (through covarites...).
% \end{itemize}
% \end{frame}


\begin{frame}
 \frametitle{Taking into account covariates}
 
 Sometimes covariates are available. They may be on:
 \begin{itemize}
  \item nodes,
  \item edges,
  \item both.
 \end{itemize}

 \bigskip
 
 
 \begin{enumerate}
  \item They can be used a posteriori to explain blocks inferred by SBM.
  \item Extension of the SBM which takes into account covariates. Blocks are structure of interaction which is not 
  explained by covariates !
 \end{enumerate}

 \bigskip
 
 If covariates are sampling conditions, case 2 may more interesting.
 
\end{frame}




\begin{frame}
 \frametitle{Probabilistic model for networks in a nutshell}
 
 SBM/LBM
 \begin{itemize}
  \item generative models,
  \item flexible,
  \item comprehensive models which can be linked to a lot of classical descriptors.
  
 \end{itemize}

 \bigskip
 Extension of the binary SBM model are quite natural:
 
 \begin{itemize}
  \item all the one presented above,
  \item missing data in the network,
  \item multi-layers ?
 \end{itemize}

 
 
\end{frame}



\end{document}
